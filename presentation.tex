\documentclass{beamer}
\usepackage{graphicx}

\usetheme{Warsaw}
\usefonttheme{professionalfonts}

\input{newcommand}

\title{Poincar\'e Embeddings}

\author{Gai Yu}

\begin{document}

\frame{\titlepage}

\begin{frame}

\frametitle{Objectives}

\begin{itemize}
\item Word embedding
\[
\call (Z) = \sum_{(u, v) \in \cald} \log \frac{\exp (-d(\bmz_u, \bmz_v))}{\sum_{w \in \caln (u, v)} \exp (-d(\bmz_u, \bmz_w))}
\]
where
\[
\caln (u, v) = \{w | (u, w) \ne \cald\} \cup \{v\}
\]
\item Node embedding
\[
\call (Z) = \sum_{(u, v) \in \cale} \log \bbp \{(u, v) \in \cale\}
\]
where
\[
\bbp \{(u, v) \in \cale\} = \frac1{\exp \left(\frac{d(\bmz_u, \bmz_v) - r}t\right) + 1}
\]
\end{itemize}

\end{frame}

\begin{frame}

\frametitle{Choices of $d(\bmx, \bmy)$}

\begin{itemize}
\item The Euclidean distance
\[
d(\bmx, \bmy) = |\bmx - \bmy|
\]
\item The hyperbolic distance
\[
d(\bmx, \bmy) = \arcosh \left(1 + \frac{2 |\bmx - \bmy|^2}{(1 - |\bmx|^2) (1 - |\bmy|^2)}\right) \qquad |\bmx|, |\bmy| < 1
\]
\end{itemize}

\end{frame}

\begin{frame}

\frametitle{Why hyperbolic distance?}

\[
d(\bmx, \bmy) = \arcosh \left(1 + \frac{2 |\bmx - \bmy|^2}{(1 - |\bmx|^2) (1 - |\bmy|^2)}\right)
\]

\[
\arcosh = \log \left(x + \sqrt{x^2 - 1}\right)
\]

\begin{figure}
\includegraphics[scale=0.5]{distance}
\caption{Growth of Poincaré distance \cite{nickel2017poincare}}
\end{figure}

\end{frame}

\begin{frame}

\frametitle{Why hyperbolic distance? (continued)}

In $n$-dimensional Euclidean space,
\[
V_n = \frac{\pi^{\frac{n}2} r^n}{\Gamma \left(1 + \frac{n}2\right)}
\]

In $(n + 1)$-dimensional hyperbolic space, an $n$-sphere of radius $r$ is isometric to a Euclidean $n$-sphere of Euclidean radius $R \sinh \frac{r}{R}$.

\end{frame}

\begin{frame}

\frametitle{Optimization: SGD on Riemannian manifold \cite{bonnabel2013stochastic}}

Update rule:
\[
\bmz_{t + 1}
= \exp_{\bmz_t} \left(-\eta_t \nabla_\bmz^R \call (\bmz)\right)
= \exp_{\bmz_t} \left(-\eta_t g_\bmz^{-1} \nabla_\bmz^E \call (\bmz)\right)
\]
where
\begin{itemize}
\item $\exp_\bmz$ is the exponential map at $\bmz$
\item $\nabla_\bmz^R$ is the Riemannian gradient at $\bmz$
\item $g_\bmz^{-1}$ is the inverse metric tensor at $\bmz$
\item $\nabla_\bmz^E$ is the Euclidean gradient at $\bmz$
\end{itemize}

The hyperbolic distance corresponds to a simple metric tensor
\[
g_\bmz = \left(\frac2{1 - |\bmz|^2}\right)^2 I
\]

\end{frame}

\begin{frame}[allowframebreaks]

\bibliographystyle{unsrt}
\bibliography{presentation}

\end{frame}

\end{document}
